
\documentclass[utf8]{frontiersSCNS} % for Science, Engineering and Humanities and Social Sciences articles
\usepackage[utf8]{inputenc}
%\usepackage{inputenc}
\usepackage[english]{babel}
\DeclareUnicodeCharacter{2264}{$\leq$}
\usepackage{letltxmacro}



\usepackage{stringenc}
\usepackage{pdfescape}

\makeatletter
\renewcommand*{\UTFviii@defined}[1]{%
  \ifx#1\relax
    \begingroup
      % Remove prefix "\u8:"
      \def\x##1:{}%
      % Extract Unicode char from command name
      % (utf8.def does not support surrogates)
      \edef\x{\expandafter\x\string#1}%
      \StringEncodingConvert\x\x{utf8}{utf16be}% convert to UTF-16BE
      % Hexadecimal representation
      \EdefEscapeHex\x\x
      % Enhanced error message
      \PackageError{inputenc}{Unicode\space char\space \string#1\space
                              (U+\x)\MessageBreak
                              not\space set\space up\space
                              for\space use\space with\space LaTeX}\@eha
    \endgroup
  \else\expandafter
    #1%
  \fi
}
\makeatother







\LetLtxMacro{\ORIGselectlanguage}{\selectlanguage}
\makeatletter
\DeclareRobustCommand{\selectlanguage}[1]{%
  \@ifundefined{alias@\string#1}
    {\ORIGselectlanguage{#1}}
    {\begingroup\edef\x{\endgroup
       \noexpand\ORIGselectlanguage{\@nameuse{alias@#1}}}\x}%
}
\newcommand{\definelanguagealias}[2]{%
  \@namedef{alias@#1}{#2}%
}
\makeatother

\definelanguagealias{eng}{english}

\makeatletter
\let\l@eng\l@english
\makeatother

%\setcitestyle{square}
\usepackage{url,hyperref,lineno,microtype}
\usepackage[onehalfspacing]{setspace}
\linenumbers


% Leave a blank line between paragraphs instead of using \\


\def\keyFont{\fontsize{8}{11}\helveticabold }
\def\firstAuthorLast{Diego Angulo {et~al.}} %use et al only if is more than 1 author
\def\Authors{Diego Andrés Angulo\,$^{1,*}$, José Tiberio Hernández\,$^{1}$ , James Oliver and Cyril Schneider\,$^3$}
% Affiliations should be keyed to the author's name with superscript numbers and be listed as follows: Laboratory, Institute, Department, Organization, City, State abbreviation (USA, Canada, Australia), and Country (without detailed address information such as city zip codes or street names).
% If one of the authors has a change of address, list the new address below the correspondence details using a superscript symbol and use the same symbol to indicate the author in the author list.
\def\Address{$^{1}$ IMAGINE, Universidad de los Andes, Bogotá , Colombia \\
$^{2}$VRAC, Iowa State University, Ames , IA, Unites States
\\
$^{3}$AXE Neuroscience, CHUL, Quebec , QC, Canada  }
% The Corresponding Author should be marked with an asterisk
% Provide the exact contact address (this time including street name and city zip code) and email of the corresponding author
\def\corrAuthor{Diego A. Angulo}
\def\corrAddress{IMAGINE, Universidad de los Andes, Carrera 1 \# 18A -12 , Bogotá, Colombia}
\def\corrEmail{da.angulo39@uniandes.edu.co}




\begin{document}
\onecolumn
\firstpage{1}

\title[Braviz]{Braviz: Visual Exploratory Analysis of Brain Datasets} 

\author[\firstAuthorLast ]{\Authors} %This field will be automatically populated
\address{} %This field will be automatically populated
\correspondance{} %This field will be automatically populated

\extraAuth{}% If there are more than 1 corresponding author, comment this line and uncomment the next one.
%\extraAuth{corresponding Author2 \\ Laboratory X2, Institute X2, Department X2, Organization X2, Street X2, City X2 , State XX2 (only USA, Canada and Australia), Zip Code2, X2 Country X2, email2@uni2.edu}


\maketitle

%%%%%%%%%%%%%%%%%%%%%%%%%%%%%%%%%%%%%%%%%%%%%%%%%%%%%%%%%%%%%%%%%%%%%%%%%%%%%%%%%%%%%%%%%%%%%%%%%%%%%%%%%%%%%%%%%%%%%%%%%%%%%%%%%%%%%%%%%%%%%%%%%%%%%%%%%%%%%%%%%%%%%%%%%%%%%%%%%%%%%%%%%%%%%%%%%%%%%%%%%%%%%%%%%%%%%%%%%%%%%%%%%%%%%%%
%%% The sections below are for reference only.
%%%
%%% For Original Research Articles, Clinical Trial Articles, and Technology Reports the section headings should be those appropriate for your field and the research itself. It is recommended to organize your manuscript in the
%%% following sections or their equivalents for your field:
%%% Abstract, Introduction, Material and Methods, Results, and Discussion.
%%% Please note that the Material and Methods section can be placed in any of the following ways: before Results, before Discussion or after Discussion.
%%%
%%%For information about Clinical Trial Registration, please go to http://www.frontiersin.org/about/AuthorGuidelines#ClinicalTrialRegistration
%%%
%%% For Clinical Case Studies the following sections are mandatory: Abstract, Introduction, Background, Discussion, and Concluding Remarks.
%%%
%%% For all other article types there are no mandatory sections.
%%%%%%%%%%%%%%%%%%%%%%%%%%%%%%%%%%%%%%%%%%%%%%%%%%%%%%%%%%%%%%%%%%%%%%%%%%%%%%%%%%%%%%%%%%%%%%%%%%%%%%%%%%%%%%%%%%%%%%%%%%%%%%%%%%%%%%%%%%%%%%%%%%%%%%%%%%%%%%%%%%%%%%%%%%%%%%%%%%%%%%%%%%%%%%%%%%%%%%%%%%%%%%%%%%%%%%%%%%%%%%%%%%%%%%%

\begin{abstract}

%%% Leave the Abstract empty if your article falls under any of the following categories: Editorial Book Review, Commentary, Field Grand Challenge, Opinion or specialty Grand Challenge.
\section{}
Brain researchers typically deal with large amounts of data from different sources and often, of different nature. This requires the use of several different software tools and makes it cumbersome and time consuming to answer simple questions. Because of this, data is not used to its fullest potential, and exploratory analysis is rarely done . This paper presents a software tool called BRAVIZ that integrates access to several data types and automates many of the cumbersome and error-prone tasks required to explore typical neuroscience data. This work focuses on integrating interactive visualization with real-time statistical analyses to facilitate exploration and discovery. BRAVIZ enables an inversion of the typical neuroscience analysis process by emphasizing images as the main organizing objects in the process rather than relying in abstract numerical indicators. This encourages researchers to notice trends and relationships, which motivate additional analyses and generally gain a fuller understanding of the phenomena represented by the data.  A case study is presented that incorporates MRI, DTI, and fMRI images together with a large amount of neuropsychological and clinical data.  The case study demonstrates how BRAVIZ enables researchers to discover new hypotheses about the relationships between structures and functions of the brain.

\tiny
 \keyFont{ \section{Keywords:} Exploratory Analysis, Visual Analytics, Brain Data, MRI, Tractography, Cohorts } %All article types: you may provide up to 8 keywords; at least 5 are mandatory.
\end{abstract}

\section{Introduction}

Visualizing and exploring data from brain studies is not an easy task. These studies include a large variety of data from each participant, some acquired from physical measurements or instrumentation, including neuroimages, while other information is acquired from cognitive or motor testing of the subjects. Demographical and clinical information from the subjects is also essential to make sense of the acquired data.
This work focuses on brain studies in which imaging scans are applied to a group of subjects. Several image modalities can be acquired during the scan session. This spatial data is complemented by scalar measures, and other clinical variables. Often the objective of the study is to search for differences in imaging data between groups of subjects, or to find relationships between structure and function of the brain. In such cases the mentioned context information become crucial for the analysis of the data. 


In this research, a particularly representative case study, the kmc project \cite{schneider_cerebral_2012}, is introduced to both motivate the requirements for BRAVIZ and demonstrate its effectiveness. The kmc study explores the effect of treatment options on premature births and data on subjects was collected from several fronts. The original randomized study involving about 750 preterm babies was conducted in 1994 \citep{charpak_kangaroo_1997}. These kids were followed during their first year \citep{charpak_randomized_2001, tessier_kangaroo_2009} of life and several clinical and socio-economical variables were registered. Forty of the kids from the original study were re located at fifteen years of age, the  subjects went through several neuropsychological tests, measuring attention, memory, reasoning skills and hand-eye coordination among others. They also went to vision, hearing and full medical examinations. Finally, they were scanned with Structural MRI, DTI and several fMRI protocols. \footnote{agregar este protocolo} These images were processed using freesurfer \citep{fischl_freesurfer_2012}, fsl\citep{jenkinson_fsl_2012}, camino\citep{cook_camino:_2006}, and spm\citep{friston_statistical_2006}; which extracted several numerical measurements and geometric structures as segmentations and tractographies. All of this data was collected for testing specific hypotheses but the specialists involved in the study are also interested in analyzing it for unexpected relationships and trends; in other words, to perform an exploratory analysis\citep{tukey_we_1980}.


Exploratory analyses require visualizing different kinds of data, and interactively moving through it, and across participants. It requires visualizing image data in context; and tracing numerical features back to the image of origin. It is necessary to have an overview of the data in order to grasp patterns and detect outliers. Outliers need to be analyzed closely to understand if they are real data or the result of some mistake (see \cite{schneiderman_designing_1998}). 

Current analysis tools  are designed to support the different steps of confirmatory analysis, and most image viewers are designed for quality control or for producing publication level images. However, exploratory research requires integrated analysis and visualization tools that provide immediate access to the data. Confirmatory analysis is a careful linear process, which starts with a hypothesis, processes data in a specific manner and performs planned statistics in order to prove the hypothesis. In contrast exploratory analysis is iterative and goes through the data multiple times in different ways. The objective is to find trends, patterns and general insights in the data, which may later become hypotheses that can be tested in a new experiment.

Visual analytics\citep{cook_illuminating_2005} is a emerging field that attempts to put human experts and machines at the same level so that, working together, data can be better understood. This requires very efficient communication between the computer and the expert, which is accomplished through rich interactive visualizations. In visual analytics the strengths of computers and human experts complement each other. Human creativity, intuition, and expertise can guide the analysis into unexpected areas. Applying the visual analytics framework to neuroscience will allow researchers to extract more information from their large, multi-source and labor-intensive datasets, and potentially lead to new discoveries.

This paper presents BRAVIZ, a software tool to facilitate visual exploration of brain data. It provides access to image data such as MRI, fMRI and DWI, as well as processed data, like surface segmentations, activation volumes and tractographies. It also supports access to tabular data and integrates some simple statistical analysis tools. In contrast to the conventional workflow that introduces visualization only at the end of the analysis, BRAVIZ positions interactive visualization at the core of exploratory research. Simple, efficient and intuitive access to visualization facilitates creation of analysis input, evaluation of analysis results, and inspiration of new research hypotheses. The system was designed following a user-centered methodology and aims to solve the bottlenecks identified in neuroscience domain experts’ workflow. The contributions of the systems can be summarized as:
\begin{itemize}
\item Visualization integrated at every step
\item "Details on demand" via integrated data linking 
\item Accelerated exploratory research by removing data access overhead
\item Support concrete domain user needs
\item Give researchers access to data that is typically out of their area of expertise, and improve communication between experts
\item Opportunistic integration of existing tools and algorithms.

\end{itemize}

%\begin{methods}
\section{Description}

Include also here the not repeated paragraphs from the Methodology (really motivation) section of the last draft.

\subsection{Example Tools}

Not sure if this needs a different section, maybe it should be summarized into the description

\section{Related Work}

\section{Implementation}

\section{Case Study}

\section{Discussion}

\section{Figures}

\section*{Acknowledgments}


\bibliographystyle{frontiersinSCNS_ENG_HUMS} % for Science, Engineering and Humanities and Social Sciences articles, for Humanities and Social Sciences articles please include page numbers in the in-text citations
%\bibliographystyle{frontiersinHLTH&FPHY} % for Health and Physics articles
\bibliography{zotero}

%%% Upload the *bib file along with the *tex file and PDF on submission if the bibliography is not in the main *tex file

\section*{Figures}

%%% Use this if adding the figures directly in the mansucript, if so, please remember to also upload the files when submitting your article
%%% There is no need for adding the file termination, as long as you indicate where the file is saved. In the examples below the files (logo1.jpg and logo2.eps) are in the Frontiers LaTeX folder
%%% If using *.tif files convert them to .jpg or .png


\end{document}
