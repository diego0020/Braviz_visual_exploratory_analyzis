
\documentclass[utf8]{frontiersSCNS} % for Science, Engineering and Humanities and Social Sciences articles
\usepackage[utf8]{inputenc}
%\usepackage{inputenc}
\usepackage[english]{babel}
\DeclareUnicodeCharacter{2264}{$\leq$}
\usepackage{letltxmacro}



\usepackage{stringenc}
\usepackage{pdfescape}

\makeatletter
\renewcommand*{\UTFviii@defined}[1]{%
  \ifx#1\relax
    \begingroup
      % Remove prefix "\u8:"
      \def\x##1:{}%
      % Extract Unicode char from command name
      % (utf8.def does not support surrogates)
      \edef\x{\expandafter\x\string#1}%
      \StringEncodingConvert\x\x{utf8}{utf16be}% convert to UTF-16BE
      % Hexadecimal representation
      \EdefEscapeHex\x\x
      % Enhanced error message
      \PackageError{inputenc}{Unicode\space char\space \string#1\space
                              (U+\x)\MessageBreak
                              not\space set\space up\space
                              for\space use\space with\space LaTeX}\@eha
    \endgroup
  \else\expandafter
    #1%
  \fi
}
\makeatother







\LetLtxMacro{\ORIGselectlanguage}{\selectlanguage}
\makeatletter
\DeclareRobustCommand{\selectlanguage}[1]{%
  \@ifundefined{alias@\string#1}
    {\ORIGselectlanguage{#1}}
    {\begingroup\edef\x{\endgroup
       \noexpand\ORIGselectlanguage{\@nameuse{alias@#1}}}\x}%
}
\newcommand{\definelanguagealias}[2]{%
  \@namedef{alias@#1}{#2}%
}
\makeatother

\definelanguagealias{eng}{english}
\definelanguagealias{en}{english}

\makeatletter
\let\l@eng\l@english
\makeatother

%\setcitestyle{square}
\usepackage{url,hyperref,lineno,microtype}
\usepackage[onehalfspacing]{setspace}
\linenumbers


% Leave a blank line between paragraphs instead of using \\


\def\keyFont{\fontsize{8}{11}\helveticabold }
\def\firstAuthorLast{Diego Angulo {et~al.}} %use et al only if is more than 1 author
\def\Authors{Diego Andrés Angulo\,$^{1,*}$, José Tiberio Hernández\,$^{1}$ , James Oliver and Cyril Schneider\,$^3$}
% Affiliations should be keyed to the author's name with superscript numbers and be listed as follows: Laboratory, Institute, Department, Organization, City, State abbreviation (USA, Canada, Australia), and Country (without detailed address information such as city zip codes or street names).
% If one of the authors has a change of address, list the new address below the correspondence details using a superscript symbol and use the same symbol to indicate the author in the author list.
\def\Address{$^{1}$ IMAGINE, Universidad de los Andes, Bogotá , Colombia \\
$^{2}$VRAC, Iowa State University, Ames , IA, Unites States
\\
$^{3}$AXE Neurosciences, CHUL, Quebec , QC, Canada  }
% The Corresponding Author should be marked with an asterisk
% Provide the exact contact address (this time including street name and city zip code) and email of the corresponding author
\def\corrAuthor{Diego A. Angulo}
\def\corrAddress{IMAGINE, Universidad de los Andes, Carrera 1 \# 18A -12 , Bogotá, Colombia}
\def\corrEmail{da.angulo39@uniandes.edu.co}




\begin{document}
\onecolumn
\firstpage{1}

\title[Braviz]{Braviz: Visual Exploratory Analysis of Brain Datasets} 

\author[\firstAuthorLast ]{\Authors} %This field will be automatically populated
\address{} %This field will be automatically populated
\correspondance{} %This field will be automatically populated

\extraAuth{}% If there are more than 1 corresponding author, comment this line and uncomment the next one.
%\extraAuth{corresponding Author2 \\ Laboratory X2, Institute X2, Department X2, Organization X2, Street X2, City X2 , State XX2 (only USA, Canada and Australia), Zip Code2, X2 Country X2, email2@uni2.edu}


\maketitle

%%%%%%%%%%%%%%%%%%%%%%%%%%%%%%%%%%%%%%%%%%%%%%%%%%%%%%%%%%%%%%%%%%%%%%%%%%%%%%%%%%%%%%%%%%%%%%%%%%%%%%%%%%%%%%%%%%%%%%%%%%%%%%%%%%%%%%%%%%%%%%%%%%%%%%%%%%%%%%%%%%%%%%%%%%%%%%%%%%%%%%%%%%%%%%%%%%%%%%%%%%%%%%%%%%%%%%%%%%%%%%%%%%%%%%%
%%% The sections below are for reference only.
%%%
%%% For Original Research Articles, Clinical Trial Articles, and Technology Reports the section headings should be those appropriate for your field and the research itself. It is recommended to organize your manuscript in the
%%% following sections or their equivalents for your field:
%%% Abstract, Introduction, Material and Methods, Results, and Discussion.
%%% Please note that the Material and Methods section can be placed in any of the following ways: before Results, before Discussion or after Discussion.
%%%
%%%For information about Clinical Trial Registration, please go to http://www.frontiersin.org/about/AuthorGuidelines#ClinicalTrialRegistration
%%%
%%% For Clinical Case Studies the following sections are mandatory: Abstract, Introduction, Background, Discussion, and Concluding Remarks.
%%%
%%% For all other article types there are no mandatory sections.
%%%%%%%%%%%%%%%%%%%%%%%%%%%%%%%%%%%%%%%%%%%%%%%%%%%%%%%%%%%%%%%%%%%%%%%%%%%%%%%%%%%%%%%%%%%%%%%%%%%%%%%%%%%%%%%%%%%%%%%%%%%%%%%%%%%%%%%%%%%%%%%%%%%%%%%%%%%%%%%%%%%%%%%%%%%%%%%%%%%%%%%%%%%%%%%%%%%%%%%%%%%%%%%%%%%%%%%%%%%%%%%%%%%%%%%

\begin{abstract}

%%% Leave the Abstract empty if your article falls under any of the following categories: Editorial Book Review, Commentary, Field Grand Challenge, Opinion or specialty Grand Challenge.
\section{}
Brain researchers typically deal with large amounts of data from different sources and often, of different nature. This requires the use of several different software tools and makes it cumbersome and time consuming to answer simple questions. Because of this, data is not used to its fullest potential, and exploratory analysis is rarely done . This paper presents a software tool called BRAVIZ that integrates access to several data types and automates many of the cumbersome and error-prone tasks required to explore typical neuroscience data. This work focuses on integrating interactive visualization with real-time statistical analyses to facilitate exploration and discovery. BRAVIZ enables an inversion of the typical neuroscience analysis process by emphasizing images as the main organizing objects in the process rather than relying in abstract numerical indicators. This encourages researchers to notice trends and relationships, which motivate additional analyses and generally gain a fuller understanding of the phenomena represented by the data.  A case study is presented that incorporates MRI, DTI, and fMRI images together with a large amount of neuro-psychological and clinical data.  The case study demonstrates how BRAVIZ enables researchers to discover new hypotheses about the relationships between structures and functions of the brain.

\tiny
 \keyFont{ \section{Keywords:} Exploratory Analysis, Visual Analytics, Brain Data, MRI, Tractography, Cohorts } %All article types: you may provide up to 8 keywords; at least 5 are mandatory.
\end{abstract}

\section{Introduction}

Visualizing and exploring data from brain studies is not an easy task. These studies include a large variety of data from each participant, some acquired from physical measurements or instrumentation, including neuroimages, while other information is acquired from cognitive or motor testing of the subjects. Demographical and clinical information from the subjects is also essential to make sense of the acquired data.
This work focuses on brain studies in which imaging scans are applied to a group of subjects. Several image modalities can be acquired during the scan session. This spatial data is complemented by scalar measures, and other clinical variables. Often the objective of the study is to search for differences in imaging data between groups of subjects, or to find relationships between structure and function of the brain. In such cases the mentioned context information become crucial for the analysis of the data. 


In this research, a particularly representative case study, the kmc project \cite{schneider_cerebral_2012}, is introduced to both motivate the requirements for BRAVIZ and demonstrate its effectiveness. The kmc study explores the effect of treatment options on premature births and data on subjects was collected from several fronts. The original randomized study involving about 750 preterm babies was conducted in 1994 \citep{charpak_kangaroo_1997}. These kids were followed during their first year \citep{charpak_randomized_2001, tessier_kangaroo_2009} of life and several clinical and socio-economical variables were registered. Forty of the kids from the original study were re located at fifteen years of age, the  subjects went through several neuro-psychological tests, measuring attention, memory, reasoning skills and hand-eye coordination among others. They also went to vision, hearing and full medical examinations. Finally, they were scanned with Structural MRI, DTI and several fMRI protocols. \footnote{agregar este protocolo} These images were processed using freesurfer \citep{fischl_freesurfer_2012}, fsl\citep{jenkinson_fsl_2012}, camino\citep{cook_camino:_2006}, and spm\citep{friston_statistical_2006}; which extracted several numerical measurements and geometric structures as segmentations and tractographies. All of this data was collected for testing specific hypotheses but the specialists involved in the study are also interested in analyzing it for unexpected relationships and trends; in other words, to perform an exploratory analysis\citep{tukey_we_1980}.


Exploratory analyses require visualizing different kinds of data, and interactively moving through it, and across participants. It requires visualizing image data in context; and tracing numerical features back to the image of origin. It is necessary to have an overview of the data in order to grasp patterns and detect outliers. Outliers need to be analyzed closely to understand if they are real data or the result of some mistake (see \cite{schneiderman_designing_1998}). 

Current analysis tools  are designed to support the different steps of confirmatory analysis, and most image viewers are designed for quality control or for producing publication level images. However, exploratory research requires integrated analysis and visualization tools that provide immediate access to the data. Confirmatory analysis is a careful linear process, which starts with a hypothesis, processes data in a specific manner and performs planned statistics in order to prove the hypothesis. In contrast exploratory analysis is iterative and goes through the data multiple times in different ways. The objective is to find trends, patterns and general insights in the data, which may later become hypotheses that can be tested in a new experiment.

Visual analytics\citep{cook_illuminating_2005} is a emerging field that attempts to put human experts and machines at the same level so that, working together, data can be better understood. This requires very efficient communication between the computer and the expert, which is accomplished through rich interactive visualizations. In visual analytics the strengths of computers and human experts complement each other. Human creativity, intuition, and expertise can guide the analysis into unexpected areas. Applying the visual analytics framework to neuroscience will allow researchers to extract more information from their large, multi-source and labor-intensive datasets, and potentially lead to new discoveries.

This paper presents BRAVIZ, a software tool to facilitate visual exploration of brain data. It provides access to image data such as MRI, fMRI and DWI, as well as processed data, like surface segmentations, activation volumes and tractographies. It also supports access to tabular data and integrates some simple statistical analysis tools. In contrast to the conventional workflow that introduces visualization only at the end of the analysis, BRAVIZ positions interactive visualization at the core of exploratory research. Simple, efficient and intuitive access to visualization facilitates creation of analysis input, evaluation of analysis results, and inspiration of new research hypotheses. The system was designed following a user-centered methodology and aims to solve the bottlenecks identified in neuroscience domain experts’ workflow. To maintain interactivity and provide high quality analysis tools, BRAVIZ leverages several state state-of-the-art open-source applications and libraries.  The contributions of the systems can be summarized as:
\begin{itemize}
\item Visualization integrated at every step
\item "Details on demand" via integrated data linking 
\item Accelerated exploratory research by removing data access overhead
\item Support concrete domain user needs
\item Give researchers access to data that is typically out of their area of expertise, and improve communication between experts
\item Opportunistic integration of existing tools and algorithms.

\end{itemize}

%\begin{methods}
\section{Description}

BRAVIZ was designed following a "user centered" approach\citep{fernandez_user-centered_2013,wassink_applying_2009}. The authors worked closely with brain researchers of several specialties, visited several labs and hospitals, and learned as much as possible about typical neuroscience research workflows. An extensive literature review in the domain led to better understanding of the kinds of data patterns neuroscience researchers find interesting. In addition, the graphics and pictures in these publications were used as models for BRAVIZ visualizations so that they would look familiar to the neuroscience research community. The design, development and evaluation process consisted of several iterative cycles, prototypes were shared with domain experts, and their feedback motivated design ideas and usability enhancements for the next generation. As the development cycles progressed, the domain experts ventured further away from common paradigms; they were encouraged to dream, and not to worry about how complex it would be to implement the ideas. 

Instead of one large application with numerous features and controls, BRAVIZ is divided into several small tools. This tools are designed to analyze cohort data, i.e., data that is composed from a set of subjects with several measures. Smaller samples of subjects, or subsamples, can also be analyzed and compared against each other. Nominal or numerical measures of the subjects are called variables, and these complement the neuro-image based spatial data. Note that the set of variables can be extended by deriving new values from spatial data or existing variables. 

BRAVIZ tools combine numerical and nominal variables with spatial data. There are tools to look at spatial data from a single subject with high detail, as well as tools to look at spatial data associated with a subsample; both with numerical and categorical data as context. Another group of tools is focused on exploring these variables, but always linking back to details of each subject.

Each of the tools has data visualization as its centerpiece. This visualization can be customized, usually by adding or removing variables or spatial data, according to the question at hand. The state of the visualization can be saved  to recall it at a future time or to share it with colleagues. 

All visualizations are interactive and capable of generating a dialogue with the user.
As Colin Ware \citep{ware_information_2004} said "The best visualizations are not static images to be printed in books, but fluid, dynamic artefacts that respond to the need for different views or for more detailed information".  

Though the system is built from several small applications, they are designed to work together synchronously in order to support more complex tasks. Subsamples, variables and subjects of interest can be shared across tools and among colleagues. 
Several of the calculations are made ahead of time and kept in a cache in order to speed-up response at analysis time. These features, combined with having all data and analysis tools a couple of clicks away; provides a pleasant and efficient exploratory analysis environment. 

A main objective of the project is fostering collaboration between experts, especially inside teams including multiple specialties. Thus the initial stages of design focused on identifying the obstacles that affected communication between experts, and looked explored ways of mitigating them. The team of experts was composed of radiologists, psychiatrists, pediatricians, physicians, physiologists, statisticians, engineers and economists. Each of these groups of experts typically utilizes tools designed to foster their analysis and contribution to the team, but very few develop skills in tools outside of their areas of expertise.

For example, several experts in neuroimaging typically use Freesurfer  \citep{fischl_freesurfer_2012}, fsl\citep{jenkinson_fsl_2012} and spm \citep{friston_statistical_2006} toolkits to analyze data. All of these sets include applications for interactive visualization. 3D slicer \citep{fedorov_3d_2012}, Brain Visa \citep{cointepas_brainvisa:_2001} and ITKSnap \citep{yushkevich_user-guided_2006} are other commonly used tool which provide advanced visualization and processing features. On the clinical side Osirix\citep{rosset_osirix:_2004} and proprietary software bundled with PACS and imaging equipment are very popular. By analyzing these tools, and the way  experts use them, the BRAVIZ team learned what was expected from image viewers, which features were important and which ones were seldom used. This effort also exposed typical process bottlenecks that affect efficient exploratory analyzes.

In statistics, data exploration tools like ggobi\citep{cook_interactive_2007}, aabel, SPSS, stata and Tableau\citep{hanrahan_tableau_2003} are used to analyze and explore datasets. Using these tools it is possible to transform data, fit models, and visualize relationships.  Interactive visualizations let the user explore complex relationships in the data, detect outliers and interesting subgroups, and grasp subtle patterns and trends. These tools implement techniques such as brushing, zooming and point identification. Insights discovered in this way can be used to develop hypotheses and tested in further experiments. 

In a typical neuroscience research group, both types of tools (visualization and statistical analysis) complement each other, and it is common for experts to use them at the same time. However moving data from one side to the other  is not trivial. While most statistical tools implement interactions that allow the user to iterate very efficiently through the data, the process for creating visualizations in the neuroimage side requires significant work; and is often focused on a single subject. Most of these tools were designed for classical hypothesis testing or case studies, and are very good for this purpose. The goal of BRAVIZ is to provide a tool that integrates the best of both worlds and better supports exploratory analyzes with spatial data. A crucial aspect is providing a direct bridge between statistics and neuroimage visualizations. By integrating concepts from both worlds BRAVIZ seeks to grasp the attention of all kinds of specialties and increase communication between them. 

%Include also here the not repeated paragraphs from the Methodology (really motivation) section of the last draft.


\subsection{Some BRAVIZ Tools}

Not sure if this needs a different section, maybe it should be summarized into the description

\section{Related Work}

In recent years the amount of MRI-Data collected has grown significantly. It is also increasingly common to find large medical image databases open for analysis, such as the human connectome project\citep{rosen_human_2010} and ADNI\citep{jack_alzheimers_2008}. This has created a need for exploration tools that will allow researchers from around the world to extract knowledge from this data. Notice that the interest here shifts from hypothesis testing towards hypothesis generation, and therefore the workflow is completely different.
  
INVIZIAN\citep{bowman_query-based_2011,bowman_feature-similarity_2012} is an application for exploring large brain datasets which integrates structural and scalar data. It organizes data in an abstract 3D space in such a way that similar brains are located close to one another. The system can display more or fewer details of each brain depending on the zoom level. It is also integrated with gGobi\citep{cook_interactive_2007} in order to explore relations between scalar values and more complex brain features. It can handle datasets of several hundreds of brains and it uses sophisticated data mining algorithms, feature extraction and abstract spaces to create the representation. The main different to our approach is that BRAVIZ focuses on leveraging the creativity, intuition, experience and knowledge of the human expert while INVIZIAN focuses on machine learning methods. In the future both approaches should be integrated.

The Visual Analytics of Brain Networks\citep{li_visual_2012} (VABN) software is focused on analyzing large connectivity datasets by defining a robust and repeatable algorithm for region of interest (ROI) placement that will act as nodes. This creates comparable networks that can be analyzed with graph based methods. The tool provides access to the underlying DWI or fMRI data in order to initially place the ROIs and to verify the results. The methods and software libraries used in this tool are very similar to BRAVIZ. However the objectives of both projects are different, as VABN does not seek to integrate connectivity data with other kinds of data. In the future VABN’s network analyzes could be integrated into BRAVIZ to further extend its capabilities.

\section{Implementation}



\section{Case Study}

\section{Discussion}

\section{Figures}

\section*{Acknowledgments}


\bibliographystyle{frontiersinSCNS_ENG_HUMS} % for Science, Engineering and Humanities and Social Sciences articles, for Humanities and Social Sciences articles please include page numbers in the in-text citations
%\bibliographystyle{frontiersinHLTH&FPHY} % for Health and Physics articles
\bibliography{zotero}

%%% Upload the *bib file along with the *tex file and PDF on submission if the bibliography is not in the main *tex file

\section*{Figures}

%%% Use this if adding the figures directly in the mansucript, if so, please remember to also upload the files when submitting your article
%%% There is no need for adding the file termination, as long as you indicate where the file is saved. In the examples below the files (logo1.jpg and logo2.eps) are in the Frontiers LaTeX folder
%%% If using *.tif files convert them to .jpg or .png


\end{document}
