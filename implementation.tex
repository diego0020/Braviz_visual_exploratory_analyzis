\section*{Implementation (Appendix)}

BRAVIZ is implemented in Python, a popular language in the neuroscience community \cite{gorgolewski_nipype:_2011}\cite{garyfallidis_dipy_2014}, which provides powerful plotting, rendering, and data manipulation libraries. The trade-off is a loss of performance compared to compiled languages, but so far interactivity has been maintained by letting VTK\cite{schroeder_vtk_1998} and Numpy\cite{van_der_walt_numpy_2011} do the most intensive operations. It has been tested in Mac, Windows and Linux.


The ``Read and Filter'' module handles all access to the file system or network. Each BRAVIZ project is composed of a SQLite database, and spatial data. This module provides other useful functions like attaching scalars to geometric objects, filtering fiber bundles and transforming between coordinate systems. Underneath it relies heavily on nibabel\cite{gorgolewski_nipype:_2011}, numpy\cite{van_der_walt_numpy_2011} , scipy\cite{jones_scipy:_2001}\cite{oliphant_python_2007} and VTK\cite{schroeder_vtk_1998} . Notice that this module handles all the necessary transformations of spatial data, and exposes a high level interface to the upper layers. In this way application designers can focus on visualization and interaction as required by end users and let the library handle registration and other low level operations. The module also provides access to numerical and categorical variables that are stored in a database. Pandas' DataFrames\cite{mckinney_data_2010} are the preferred way to manipulate this data. The database also holds annotations, scenarios, user defined geometrical objects and analysis history. 

Specific project reader modules can be created to adjust to the layout, file names and other peculiarities of each research project. The active project, favorite variables and favorite subject and other preferences are specified through a configuration file.
By accessing data through this module, developers can create applications that can be reused in several projects. 

The ``visualization'' module provides reusable components for visualizing spatial and non-spatial data. VTK is used for all the 3D visualization while 2D visualization is achieved through matplotlib \cite{hunter_matplotlib:_2007} and seaborn\cite{michael_waskom_seaborn:_2014} in native applications, or through D3 \cite{bostock_d3_2011} in web applications.

The ``interaction'' module contains, among others, several reusable user interface components, functions for doing statistical calculations in R\cite{team_r:_2012}\cite{gautier_rpy2:_2008} and additional processing for geometric data, and connection mechanisms between applications. The current version uses PyQt4 for the user interface with some web based visualizations built using the tornado framework \cite{server_source_2008}. 

Individual applications use a different mixture of the provided modules depending on user requirements. Most of the code inside each application should be about user interaction and coordination between visualizations. If new data processing algorithms or visualizations are required they should be implemented as reusable components in the library so that future applications can benefit from them. The project is licensed under a lesser general public license (LGPL) and its source code is available at \url{https://bitbucket.org/dieg0020/braviz}, together with installation instructions. Additional documentation is available at \url{http://diego0020.github.io/braviz}.